% Options for packages loaded elsewhere
\PassOptionsToPackage{unicode}{hyperref}
\PassOptionsToPackage{hyphens}{url}
%
\documentclass[
]{article}
\usepackage{lmodern}
\usepackage{amssymb,amsmath}
\usepackage{ifxetex,ifluatex}
\ifnum 0\ifxetex 1\fi\ifluatex 1\fi=0 % if pdftex
  \usepackage[T1]{fontenc}
  \usepackage[utf8]{inputenc}
  \usepackage{textcomp} % provide euro and other symbols
\else % if luatex or xetex
  \usepackage{unicode-math}
  \defaultfontfeatures{Scale=MatchLowercase}
  \defaultfontfeatures[\rmfamily]{Ligatures=TeX,Scale=1}
\fi
% Use upquote if available, for straight quotes in verbatim environments
\IfFileExists{upquote.sty}{\usepackage{upquote}}{}
\IfFileExists{microtype.sty}{% use microtype if available
  \usepackage[]{microtype}
  \UseMicrotypeSet[protrusion]{basicmath} % disable protrusion for tt fonts
}{}
\makeatletter
\@ifundefined{KOMAClassName}{% if non-KOMA class
  \IfFileExists{parskip.sty}{%
    \usepackage{parskip}
  }{% else
    \setlength{\parindent}{0pt}
    \setlength{\parskip}{6pt plus 2pt minus 1pt}}
}{% if KOMA class
  \KOMAoptions{parskip=half}}
\makeatother
\usepackage{xcolor}
\IfFileExists{xurl.sty}{\usepackage{xurl}}{} % add URL line breaks if available
\IfFileExists{bookmark.sty}{\usepackage{bookmark}}{\usepackage{hyperref}}
\hypersetup{
  pdftitle={Laboratorio 1},
  hidelinks,
  pdfcreator={LaTeX via pandoc}}
\urlstyle{same} % disable monospaced font for URLs
\usepackage[margin=1in]{geometry}
\usepackage{color}
\usepackage{fancyvrb}
\newcommand{\VerbBar}{|}
\newcommand{\VERB}{\Verb[commandchars=\\\{\}]}
\DefineVerbatimEnvironment{Highlighting}{Verbatim}{commandchars=\\\{\}}
% Add ',fontsize=\small' for more characters per line
\usepackage{framed}
\definecolor{shadecolor}{RGB}{248,248,248}
\newenvironment{Shaded}{\begin{snugshade}}{\end{snugshade}}
\newcommand{\AlertTok}[1]{\textcolor[rgb]{0.94,0.16,0.16}{#1}}
\newcommand{\AnnotationTok}[1]{\textcolor[rgb]{0.56,0.35,0.01}{\textbf{\textit{#1}}}}
\newcommand{\AttributeTok}[1]{\textcolor[rgb]{0.77,0.63,0.00}{#1}}
\newcommand{\BaseNTok}[1]{\textcolor[rgb]{0.00,0.00,0.81}{#1}}
\newcommand{\BuiltInTok}[1]{#1}
\newcommand{\CharTok}[1]{\textcolor[rgb]{0.31,0.60,0.02}{#1}}
\newcommand{\CommentTok}[1]{\textcolor[rgb]{0.56,0.35,0.01}{\textit{#1}}}
\newcommand{\CommentVarTok}[1]{\textcolor[rgb]{0.56,0.35,0.01}{\textbf{\textit{#1}}}}
\newcommand{\ConstantTok}[1]{\textcolor[rgb]{0.00,0.00,0.00}{#1}}
\newcommand{\ControlFlowTok}[1]{\textcolor[rgb]{0.13,0.29,0.53}{\textbf{#1}}}
\newcommand{\DataTypeTok}[1]{\textcolor[rgb]{0.13,0.29,0.53}{#1}}
\newcommand{\DecValTok}[1]{\textcolor[rgb]{0.00,0.00,0.81}{#1}}
\newcommand{\DocumentationTok}[1]{\textcolor[rgb]{0.56,0.35,0.01}{\textbf{\textit{#1}}}}
\newcommand{\ErrorTok}[1]{\textcolor[rgb]{0.64,0.00,0.00}{\textbf{#1}}}
\newcommand{\ExtensionTok}[1]{#1}
\newcommand{\FloatTok}[1]{\textcolor[rgb]{0.00,0.00,0.81}{#1}}
\newcommand{\FunctionTok}[1]{\textcolor[rgb]{0.00,0.00,0.00}{#1}}
\newcommand{\ImportTok}[1]{#1}
\newcommand{\InformationTok}[1]{\textcolor[rgb]{0.56,0.35,0.01}{\textbf{\textit{#1}}}}
\newcommand{\KeywordTok}[1]{\textcolor[rgb]{0.13,0.29,0.53}{\textbf{#1}}}
\newcommand{\NormalTok}[1]{#1}
\newcommand{\OperatorTok}[1]{\textcolor[rgb]{0.81,0.36,0.00}{\textbf{#1}}}
\newcommand{\OtherTok}[1]{\textcolor[rgb]{0.56,0.35,0.01}{#1}}
\newcommand{\PreprocessorTok}[1]{\textcolor[rgb]{0.56,0.35,0.01}{\textit{#1}}}
\newcommand{\RegionMarkerTok}[1]{#1}
\newcommand{\SpecialCharTok}[1]{\textcolor[rgb]{0.00,0.00,0.00}{#1}}
\newcommand{\SpecialStringTok}[1]{\textcolor[rgb]{0.31,0.60,0.02}{#1}}
\newcommand{\StringTok}[1]{\textcolor[rgb]{0.31,0.60,0.02}{#1}}
\newcommand{\VariableTok}[1]{\textcolor[rgb]{0.00,0.00,0.00}{#1}}
\newcommand{\VerbatimStringTok}[1]{\textcolor[rgb]{0.31,0.60,0.02}{#1}}
\newcommand{\WarningTok}[1]{\textcolor[rgb]{0.56,0.35,0.01}{\textbf{\textit{#1}}}}
\usepackage{graphicx,grffile}
\makeatletter
\def\maxwidth{\ifdim\Gin@nat@width>\linewidth\linewidth\else\Gin@nat@width\fi}
\def\maxheight{\ifdim\Gin@nat@height>\textheight\textheight\else\Gin@nat@height\fi}
\makeatother
% Scale images if necessary, so that they will not overflow the page
% margins by default, and it is still possible to overwrite the defaults
% using explicit options in \includegraphics[width, height, ...]{}
\setkeys{Gin}{width=\maxwidth,height=\maxheight,keepaspectratio}
% Set default figure placement to htbp
\makeatletter
\def\fps@figure{htbp}
\makeatother
\setlength{\emergencystretch}{3em} % prevent overfull lines
\providecommand{\tightlist}{%
  \setlength{\itemsep}{0pt}\setlength{\parskip}{0pt}}
\setcounter{secnumdepth}{-\maxdimen} % remove section numbering

\title{Laboratorio 1}
\author{}
\date{\vspace{-2.5em}}

\begin{document}
\maketitle

\#rmarkdown::github\_document

\begin{Shaded}
\begin{Highlighting}[]
\KeywordTok{library}\NormalTok{(readxl)}
\KeywordTok{library}\NormalTok{(readr)}
\end{Highlighting}
\end{Shaded}

\hypertarget{problema-1}{%
\section{Problema \#1}\label{problema-1}}

Creamos una lista de los archivos por cargar.

\begin{Shaded}
\begin{Highlighting}[]
\NormalTok{files =}\StringTok{ }\KeywordTok{list.files}\NormalTok{(}\DataTypeTok{pattern =} \StringTok{".xlsx"}\NormalTok{)}
\end{Highlighting}
\end{Shaded}

Creamos una función que cargue los archivos y seleccione las columnas
que necesitamos.

\begin{Shaded}
\begin{Highlighting}[]
\NormalTok{read_excel_files =}\StringTok{ }\ControlFlowTok{function}\NormalTok{(path,cols)\{}

  \CommentTok{# Leemos el archivo}
\NormalTok{  df =}\StringTok{ }\KeywordTok{read_excel}\NormalTok{(path)}
  
  \CommentTok{# Creamos las columnas de mes y año}
\NormalTok{  fecha =}\StringTok{ }\KeywordTok{strsplit}\NormalTok{(path,}\StringTok{".xlsx"}\NormalTok{)[[}\DecValTok{1}\NormalTok{]]}
\NormalTok{  df[}\StringTok{"FECHA"}\NormalTok{] =}\StringTok{ }\NormalTok{fecha}
  
\NormalTok{  sliced_df =}\StringTok{ }\NormalTok{df[cols]}
  \KeywordTok{print}\NormalTok{(}\KeywordTok{nrow}\NormalTok{(sliced_df))}
  \KeywordTok{return}\NormalTok{(sliced_df)}
\NormalTok{\}}
\end{Highlighting}
\end{Shaded}

Ejecutamos la función y guardamos el output a una \emph{lista}.

\begin{Shaded}
\begin{Highlighting}[]
\NormalTok{columnas =}\StringTok{ }\KeywordTok{c}\NormalTok{(}\StringTok{"COD_VIAJE"}\NormalTok{,}\StringTok{"CLIENTE"}\NormalTok{,}\StringTok{"UBICACION"}\NormalTok{,}\StringTok{"CANTIDAD"}\NormalTok{,}
             \StringTok{"PILOTO"}\NormalTok{,}\StringTok{"Q"}\NormalTok{,}\StringTok{"CREDITO"}\NormalTok{,}\StringTok{"UNIDAD"}\NormalTok{,}\StringTok{"FECHA"}\NormalTok{)}

\NormalTok{df_list =}\StringTok{ }\KeywordTok{lapply}\NormalTok{(files, read_excel_files,columnas)}
\end{Highlighting}
\end{Shaded}

Unimos los archivos a una sola tabla y la guardamos como \emph{csv}.

\begin{Shaded}
\begin{Highlighting}[]
\CommentTok{# Tabla final}
\NormalTok{concat_df =}\StringTok{ }\KeywordTok{do.call}\NormalTok{(rbind,df_list)}
\KeywordTok{write_excel_csv}\NormalTok{(concat_df,}\StringTok{"entregas_2018.csv"}\NormalTok{)}
\end{Highlighting}
\end{Shaded}

\hypertarget{problema-2}{%
\section{Problema \#2}\label{problema-2}}

Creamos una funcion que calcule la moda de una lista (R no tiene un
función que calcule la moda). Si existen multiples modas entonces
devulve varias modas. Si no existen modas devuelve un valor \emph{NULL}.

\begin{Shaded}
\begin{Highlighting}[]
\NormalTok{moda <-}\StringTok{ }\ControlFlowTok{function}\NormalTok{(lista)\{}
\NormalTok{  tbl =}\StringTok{ }\KeywordTok{table}\NormalTok{(lista)}
\NormalTok{  cond =}\StringTok{ }\NormalTok{(tbl}\OperatorTok{>}\DecValTok{1}\NormalTok{)}\OperatorTok{&}\NormalTok{(tbl}\OperatorTok{==}\KeywordTok{max}\NormalTok{(tbl))}
\NormalTok{  modas =}\StringTok{ }\KeywordTok{names}\NormalTok{(tbl)[cond]}
  
  \ControlFlowTok{if}\NormalTok{(}\KeywordTok{length}\NormalTok{(modas)}\OperatorTok{==}\DecValTok{0}\NormalTok{) modas=}\OtherTok{NULL}
  
  \KeywordTok{return}\NormalTok{(modas)}
\NormalTok{\}}
\end{Highlighting}
\end{Shaded}

Creamos una lista de vectores para probar la función.

\begin{Shaded}
\begin{Highlighting}[]
\NormalTok{lista_vec =}\StringTok{ }\KeywordTok{list}\NormalTok{(}\KeywordTok{c}\NormalTok{(}\DecValTok{1}\NormalTok{,}\DecValTok{2}\NormalTok{,}\DecValTok{3}\NormalTok{),}\KeywordTok{c}\NormalTok{(}\DecValTok{1}\NormalTok{,}\DecValTok{2}\NormalTok{,}\DecValTok{2}\NormalTok{,}\DecValTok{2}\NormalTok{,}\DecValTok{3}\NormalTok{),}\KeywordTok{c}\NormalTok{(}\DecValTok{1}\NormalTok{,}\DecValTok{2}\NormalTok{,}\DecValTok{2}\NormalTok{,}\DecValTok{3}\NormalTok{,}\DecValTok{3}\NormalTok{))}
\KeywordTok{lapply}\NormalTok{(lista_vec,moda)}
\end{Highlighting}
\end{Shaded}

\begin{verbatim}
## [[1]]
## NULL
## 
## [[2]]
## [1] "2"
## 
## [[3]]
## [1] "2" "3"
\end{verbatim}

\hypertarget{problema-3}{%
\section{Problema \#3}\label{problema-3}}

Cargamos el archivo de la pagina de la SAT usando la funcion
\emph{read\_delim}

\begin{Shaded}
\begin{Highlighting}[]
\NormalTok{PATH =}\StringTok{ "C:/Users/Jose/Documents/UFM/4th_year/2ndo_Semestre/Data Wrangling/Data/datos_sat/INE_PARQUE_VEHICULAR_080720.txt"} \CommentTok{# Path en mi maquina}
\NormalTok{data_sat =}\StringTok{ }\KeywordTok{read_delim}\NormalTok{(PATH,}\DataTypeTok{delim =}\StringTok{"|"}\NormalTok{) }\CommentTok{# Columna X11 parseada por error (prob. un simbolo "|" extra)}
\end{Highlighting}
\end{Shaded}

\begin{verbatim}
## Parsed with column specification:
## cols(
##   ANIO_ALZA = col_double(),
##   MES = col_character(),
##   NOMBRE_DEPARTAMENTO = col_character(),
##   NOMBRE_MUNICIPIO = col_character(),
##   MODELO_VEHICULO = col_character(),
##   LINEA_VEHICULO = col_character(),
##   TIPO_VEHICULO = col_character(),
##   USO_VEHICULO = col_character(),
##   MARCA_VEHICULO = col_character(),
##   CANTIDAD = col_double(),
##   X11 = col_character()
## )
\end{verbatim}

\end{document}
